\documentclass[english,11pt]{article}
\usepackage{babel}
\usepackage[utf8]{inputenc}
\usepackage{blindtext}
\usepackage{textcomp} 
\usepackage[dvipsnames]{xcolor}
\usepackage{graphicx}
\usepackage{geometry}
 \geometry{
 a4paper,
 right=32mm,
 bottom=19mm,
 left=32mm,
 top=19mm,
}
\usepackage{amsmath,amsfonts,amssymb,amsthm,epsfig,epstopdf,titling,url,array}
\usepackage{physics}
\usepackage{color}
\usepackage{braket}
\usepackage{dsfont}
\usepackage{mathrsfs}
\usepackage[colorlinks,allcolors=blue]{hyperref}
\theoremstyle{definition}
\newtheorem{defn}{Definición}[section]
\newtheorem{conj}{Conjecture}[section]
\newtheorem{exmp}{Example}[section]
\usepackage{float}
\usepackage{wrapfig}
\theoremstyle{plain}
\newtheorem{thm}{Teorema}[section]
\newtheorem{lem}[thm]{Lema}
\newtheorem{prop}[thm]{Proposition}
\newtheorem*{cor}{Corollary}

\begin{document}
\title{General Information}
\maketitle
In this document we can write the rules and the statutes of the group. Given the current composition of the group (graduate and undergraduate students of Physics, Math and closely related fields) I have thought of the following setting in order to make the group work.
\par We are all very busy students, with high workloads and not too much free time, so it is more than likely that this group is going to fail on its initial purpose. My idea for making this group succeed is to make of the experience a cooperative videogame, with lives and deadlines (yeah, I don't like deadlines either, although I consider that they are necessary, at least at the beginning). 

\section*{The game}
The main objective of the game is (surprise): to solve problems. However, it is widely known that a nice story to unveil through the course of the game always motivate the players to keep playing. So we are going to have a story! From each problem we solve a new fragment of a story will be revealed. I'm not a skilled writer, so I picked a famous short story from Internet (please don't google it! That's cheating and cheating kills all the fun!).
\subsection*{Rules}
The rules of the game are the following:
\begin{itemize}
\item \textbf{“All for one and one for all.” 
― Alexandre Dumas, \textit{The Three Musketeers}} \\We are a team! The game is fundamentally collaborative. There are no individual leaderboards or anything similar.
\item \textbf{"Our lives are not in the lap of the gods, but in the lap of our cooks."― Lin Yutang} \\
The team starts with 3 lives and 6 shields.
\begin{itemize}
\item  When a problem is released the team has one week to solve it. If the problem is not solved within one week the team will lose 1 shield. The team will have an extra week to solve the problem. If the team fails again the team will lose 1 life. 
\item If the team does not have any shield, instead of a shield, the team will lose a life.
\item If a week passes without the team solving the problem and there is no shield or life left, the team dies and the game is over.
\item Each time the team solves three problems the team recovers a shield (up to a maximum of 6).
\item There is no way to recover a life.
\end{itemize}
\item \textbf{"No man left behind." ― Popular military quote.}\\
\begin{itemize}

 \item A problem is considered solved if and only if every member of the team agrees on that the problem is solved. 
 \item If a member of the team does not express any statement about the state of the resolution of the problem it will be considered that such member agrees on that the problem is solved.
 \item A problem is considered failed if the team loses a life.
 \item When a problem is either failed or solved a new problem is released.
\end{itemize}
\item \textbf{“A story has no beginning or end: arbitrarily one chooses that moment of experience from which to look back or from which to look ahead.” 
― Graham Greene, \textit{The End of the Affair}}
\\
The team wins the game if the last fragment of the story is released.
\item \textbf{“It is the greatest truth of our age: Information is not knowledge.” 
― Caleb Carr } \\
There are no limits on the tools that the team can use in order to solve the problems. Everything is allowed.
\end{itemize}

\section*{The problems}
\par The system for choosing and releasing problems and the method of doing it is open to discussion. But maybe to keep a narrative and a to adjust the curve of difficulty I should be the person who releases the problems, also being able to accept problems proposed from others members of the group. We are all students with a decent background on mathematics and physics, so we can solve interesting and difficult problems. However, the topic of a problem shouldn't be too specialized or too difficult, in order to maintain the highest number of people engaged.
\\
Examples:
\begin{itemize}

\item \textbf{Consider a particle (not necessarily on a geodesic) that has fallen inside the event horizon of a neutral non-rotating black hole $r<r_s=2M$. Calculate the maximum lifetime for a particle along a trajectory from $r=2M$ to $r=0$.}
\\ 
This problem, whether is not very difficult to solve, needs a strong background on advanced topics of physics like differential geometry or General Relativity. We will leave most of people out of the game, and we don't want that.

\item \textbf{Prove that if $n$ is an integer and $n>2$ there exist no $a,b$ and $c$ positive integers such that $a^n+b^n=c^n$ (Fermat's Last Theorem)}
\\
In this problem the opposite happens. While the problem doesn't require too much background to understand it, it took 358 years to the humanity to solve it.
\end{itemize}

\section*{Active and spectator roles}
\par We do not have to forget that the purpose of this project is to have fun solving problems and discussing them with other motivated people. For succeeding on this task some level of commitment is required. However nobody wants of this game another compulsory workload over our backs with deadlines and stressful moments. So this is the reason why there are two different status that a member of the team can choose: 
\begin{itemize}
\item \textbf{Active player:} a player that is willing to actively participate solving problems and discussing them. 
\item  \textbf{Spectator player:} a player that, whether is interested in the ongoing discussion, does not have the intention to participate actively in the solving and discussion of the problems.
\end{itemize}
\par These status are NOT permanent! All the members of the team can freely change their status according to their availability. For example: a member is starting an examination session of one month long and is not going to be able to participate actively in the group that month. S/he can write in the Whatsapp group "Hey, I'm going to be an spectator this month!" and I'll update this sheet with the current status.
\par But remember, the more active players the more likely the success.

\end{document}